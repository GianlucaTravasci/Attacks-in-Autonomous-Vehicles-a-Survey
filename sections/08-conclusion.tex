\section{Conclusion}
    In this paper, an analysis of Autonomous Vehicles state-of-art was assembled. A brief overview of information security principles and AV technology was offered. Most of the AV attacks discovered were described. 
    \newline
    Table 1 give us an overview about all the attacks that can be done on AVs except those of ML on DNN. ML on DNN attacks are new and, in my knowledge, no valid countermeasures has been found in order to make AVs really resilient and safe against them. What we notice, writing this paper, is that researchers are focusing a lot on this topic in the last years.
    \newline
    Other attacks are categorized by data authenticity, availability, integrity and confidentiality. For each attack a possible countermeasure has been proposed and a personal classification has been done. The classification aim to give a better overview to the work that has been done and the one that has to be done. If an attack is marked as "Fully mitigated" this means the a countermeasure has been found for all the possible scenario. In the other hand if an attack is marked as "partially mitigated" means that few countermeasures has been found but not for every scenario and last if an attack is "uncovered" means that a lot of work as to be done or the countermeasures proposed are not really effective. 
    \newline
    This paper highlighted the lack of security on AV finding unanimous proof about its initial claims. The future of AV cannot ignore security: threats are far too real and potentially devastating. Basic vulnerabilities needs to be promptly fixed and encryption needs to be implemented during the next years to come. 


